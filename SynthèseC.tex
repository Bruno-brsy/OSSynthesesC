\documentclass[a4paper]{article}
  \usepackage[utf8]{inputenc}
  \usepackage[cyr]{aeguill}
  \usepackage[english, francais]{babel}
  \usepackage{amsmath}
  \usepackage{layout}
  \usepackage{hyperref}
  \usepackage{listings}
  \usepackage{color}

\usepackage[top=2cm, bottom=2.5cm, left=2.5cm, right=2cm]{geometry}
\title{OS: Synthèse C}
\author{Bruno Sylin}
\date\today
\begin{document}

\definecolor{mygreen}{rgb}{0,0.6,0}
\definecolor{mygray}{rgb}{0.5,0.5,0.5}
\definecolor{mymauve}{rgb}{0.58,0,0.82}

\lstset{
literate={á}{{\'a}}1 {é}{{\'e}}1 {í}{{\'i}}1 {ó}{{\'o}}1 {ú}{{\'u}}1 {Á}{{\'A}}1 {É}{{\'E}}1 {Í}{{\'I}}1 {Ó}{{\'O}}1 {Ú}{{\'U}}1 {à}{{\`a}}1 {è}{{\`e}}1 {ì}{{\`i}}1 {ò}{{\`o}}1 {ù}{{\`u}}1 {À}{{\`A}}1 {È}{{\'E}}1 {Ì}{{\`I}}1 {Ò}{{\`O}}1 {Ù}{{\`U}}1 {ä}{{\"a}}1 {ë}{{\"e}}1 {ï}{{\"i}}1 {ö}{{\"o}}1 {ü}{{\"u}}1 {Ä}{{\"A}}1 {Ë}{{\"E}}1 {Ï}{{\"I}}1 {Ö}{{\"O}}1 {Ü}{{\"U}}1 {â}{{\^a}}1 {ê}{{\^e}}1 {î}{{\^i}}1 {ô}{{\^o}}1 {û}{{\^u}}1 {Â}{{\^A}}1 {Ê}{{\^E}}1 {Î}{{\^I}}1 {Ô}{{\^O}}1 {Û}{{\^U}}1 {œ}{{\oe}}1 {Œ}{{\OE}}1 {æ}{{\ae}}1 {Æ}{{\AE}}1 {ß}{{\ss}}1 {ű}{{\H{u}}}1 {Ű}{{\H{U}}}1 {ő}{{\H{o}}}1 {Ő}{{\H{O}}}1 {ç}{{\c c}}1 {Ç}{{\c C}}1 {ø}{{\o}}1 {å}{{\r a}}1 {Å}{{\r A}}1 {€}{{\euro}}1 {£}{{\pounds}}1 {«}{{\guillemotleft{}}}1 {»}{{\guillemotright{}}}1 {ñ}{{\~n}}1 {Ñ}{{\~N}}1 {¿}{{?`}}1,
language=C,
morekeywords={size_t}
}


% \lstset{language=c}
\maketitle
\tableofcontents
% je mets des TODO quand j'ai encore quelque chose à faire, car dans mon éditeur de texte je les vois très facilement, si vous n'avez pas fini une section, je vous invite a faire de même, ou alors écrire autre chose, mais que ça puisse se voir. Si vous voulez des choses en plus, je vous invite également à les mettre dans la section TODO, ce sera plus simple.
\section{Intro}
Programme minimal:
\begin{lstlisting}
int main (void)
{
   return 0;
}
\end{lstlisting}
Pour afficher quelque chose sur la sortie standard (souvent l'écran), on peut utiliser la fonction $puts$ qui prend comme paramètre un pointeur vers une chaîne de caractère:
\begin{lstlisting}
#include <stdio.h>

int puts (char const *);
\end{lstlisting}
Les différents types et objets possibles + glossaire:
\begin{description}
  \item [Bit] Le bit (BInary digiT) ou chiffre binaire est l'unité de mesure d'information. Il peut prendre 2 valeurs: 0 ou 1.
  \item [Byte] Le byte (ou multiplet) est le plus petit objet adressable pour une implémentation donnée. En langage C, il fait au moins huit bits.
  \item [Caractère] Un caractère est une valeur numérique qui représente un glyphe visible ('A', '5', '*' etc.) ou non (CR, LF etc.). Un caractère est de type int. Cependant, sa valeur tient obligatoirement dans un type char.
  \item [Chaîne de caractères] Une chaîne de caractères est une séquence de caractères encadrée de double quotes. \newline
  \begin{lstlisting}
    "Hello world!"
  \end{lstlisting}
  La représentation interne d'une chaîne de caractères est spécifiée. C'est un tableau de char terminé par un \guillemotleft{}\textbackslash0\guillemotright{}, le caractère de fin de chaine.
  \begin{lstlisting}
    char s[] = "Hello";
  \end{lstlisting}

  est équivalent à:
  \begin{lstlisting}
    char s[] = {'H','e','l','l','o','\0'}; // ou alors on met un 0 à la place de '\0'
  \end{lstlisting}

  \item [\guillemotleft{}Suite d'objets\guillemotright{}]
  % vu qu'on est en c, on parle vraiment d'une suite d'objets ?
  \item [Tableau] Un tableau est une suite d'objets identiques consécutifs.\newline
  Exemple:
  \begin{lstlisting}
  #include <stddef.h>
    size_t size_of_tab = sizeof (long); // type size_t
    long tab[5] = {12.34, 56.78};
    size_t size_of_tab = sizeof tab / sizeof tab[0];
    // donne le nombre d'éléments dans le tableau tab
  \end{lstlisting}

  Pour un tableau à 2 dimensions:
  \begin{lstlisting}
  p = (int *)t;  // pas sur a 100 %
  <type> t[N][M];
  t[i][j] = *(p + N*i + j) /* ou encore p[N*i + j] */
  // permet de parcourir le tableau
  \end{lstlisting}

  \item [Types:]  valeur minimal / valeur maximal les termes entre \guillemotleft{} () \guillemotright{} sont facultatifs.
  \item [char] 	0 	127
  \item [unsigned char] 	0 	255
  \item [signed char] 	-127 	127
  \item [(signed) short (int)] 	-32767 	32767
  \item [unsigned short (int)]	0 	65535
  \item [(signed) int] 	-32767 	32767
  \item [unsigned (int)] 	0 	65535
  \item [(signed) long (int)] 	-2147483647 	2147483647
  \item [unsigned long (int)] 	0 	4294967295
\end{description}
\subsubsection{Le mot-clé const}
Il permet de déclarer une variable constante par exemple:
\begin{lstlisting}
  const int n = 10;
\end{lstlisting}
\subsection{printf}
Formateur de base 	Rôle 	Type attendu 	Types compatibles
\begin{description}
  \item  [\%] 	affiche le glyphe du caractère \% 	Type attendu: int 	Types compatibles: short, char
  \item  [c]	affiche le glyphe d'un caractère imprimable 	Type attendu: int 	Types compatibles: short, char
  \item  [d]	affiche la valeur d'un entier en décimal 	Type attendu: int 	Types compatibles: short, char
  \item  [f]	affiche la valeur d'un réel en décimal avec virgule fixe 	Type attendu: double Type compatibles: float
  \item  [s]	affiche les glyphes d'une chaîne de caractères 	Type attendu: char *
\end{description}
% pour le programme exemple ci-dessous ce serait pas mieux de mettre à côté des instructions l'output attendu en commentaire ? (gain de place)
\begin{lstlisting}
#include <stdio.h>

int main (void)
{
   printf ("%c\n", 'A'); // Affiche "A" puis retour à la ligne
   printf ("%d\n", 123); // Affiche "123" puis retour à la ligne
   printf ("%d, '%c'\n", 'A', 'A'); // Affiche "65", "A" puis retour à la ligne

   printf ("Hello world\n"); // Affiche "Hello world" puis retour à la ligne
   printf ("Hello %s\n", "world"); // idem
   printf ("%s\n", "Hello world"); // idem

   return 0;
}
\end{lstlisting}
\subsection{Fonction}
Une fonction est une séquence d'instructions dans un bloc nommé. Une fonction est constituée de la séquence suivante:
\begin{description}
  \item type (ou le mot clé void)
  \item identificateur (le nom de la fonction, ici main)
  \item parenthèse ouvrante \guillemotleft{} (\guillemotright{}
  \item liste de paramètres ou une liste vide (dans ce cas, on écrit void)
  \item parenthèse fermante \guillemotleft{}) \guillemotright{}
  \item accolade ouvrante \guillemotleft{}\{\guillemotright{}
  \item liste des instructions terminées par un point virgule \guillemotleft{};\guillemotright{}, ou rien
  \item accolade fermante \guillemotleft{}\}\guillemotright{}.
\end{description}
  L'utilisateur peut créer ses propres fonctions. Une fonction peut appeler une autre fonction. Généralement, une fonction réalisera une opération bien précise. L'organisation hiérarchique des fonctions permet un raffinement en partant du niveau le plus élevé (main()) en allant au niveau le plus élémentaire (atomique).\newline
On gardera cependant en tête que la multiplication des niveaux introduit une augmentation de la taille du code et du temps de traitement.\newline
Une fonction ne peut être appelée qu'à partir d'une autre fonction.\newline
Il faut savoir qu'en C, un paramètre ne fait que transmettre une valeur. Modifier la valeur d'un paramètre n'aura jamais d'effet sur la valeur originale:
\begin{lstlisting}
#include <stdio.h>

void fonction (int x)
{
  printf ("x = %d (dans la fonction)\n", x); /* x = 123 */

  x = 456;
  printf ("x = %d (dans la fonction)\n", x); /* x = 456 */
}

int main (void)
{
  int a = 123;

  printf ("a = %d\n", a); /* a = 123 */

  fonction (a);
  printf ("a = %d\n", a); /* a = 123 */
  return 0;
}
\end{lstlisting}
\subsubsection{Déclaration / implémentation}
Comme en C++\newline
Exemple de déclaration
\begin{lstlisting}
  int function (void);
\end{lstlisting}
\subsubsection{Les macros}
\begin{lstlisting}
  #define <macro> <le texte de remplacement>
\end{lstlisting}
Peut également être utilisé comme petite fonction, par exemple:
\begin{lstlisting}
  #define carre(x) ((x) * (x))
  carre(3)  // sera remplacé par 3 * 3
\end{lstlisting}
\subsubsection{Typedef}
Le C dispose d'un mécanisme très puissant permettant au programmeur de créer de nouveaux types de données en utilisant le mot clé typedef. Par exemple:
\begin{lstlisting}
  typedef int ENTIER;
  ENTIER a, b;
\end{lstlisting}
Définit le type ENTIER comme n'étant autre que le type int. Bien que dans ce cas, un simple \#define aurait pu suffire, il est toujours recommandé d'utiliser typedef qui est beaucoup plus sûr.

\subsection{Compilation}
TODO:
\subsection{Les pointeurs}
Le langage C dispose d'un opérateur \guillemotleft{}\&\guillemotright{} permettant de récupérer l'adresse en mémoire d'une variable ou d'une fonction quelconque. Par exemple, si n est une variable, \&n désigne l'adresse de n. \newline
Le C dispose également d'un opérateur \guillemotleft{}*\guillemotright{} permettant d'accéder au contenu de la mémoire dont l'adresse est donnée. Par exemple, supposons qu'on ait:
\begin{lstlisting}
  n = 10;
  // est tout a fait identique à
  * ( &n ) = 10;
\end{lstlisting}
Les pointeurs en langage C sont typés et obéissent à l'arithmétique des pointeurs que nous verrons un peu plus loin. Supposons que l'on veuille créer une variable \guillemotleft{} p \guillemotright{} destinée à recevoir l'adresse d'une variable de type int. \guillemotleft{} p \guillemotright{} s'utilisera alors de la façon suivante:
\begin{lstlisting}
  int n;
  int * p;
  p = &n;
  *p = 5;
\end{lstlisting}
Exemple dans une fonction qui permute deux variables:
\begin{lstlisting}
#include <stdio.h>

void permuter(int * addr_a, int * addr_b);

int main(){
    int a = 10, b = 20;

    permuter(&a, &b);
    printf("a = %d\nb = %d\n", a, b);

    return 0;
}

void permuter(int * addr_a , int * addr_b)
/***************\
* addr_a <-- &a *
* addr_b <-- &b *
\***************/
{
    int c;

    c = *addr_a;
    *addr_a = *addr_b;
    *addr_b = c;
}
\end{lstlisting}
\subsubsection{Les pointeurs générique}
 Le type des pointeurs génériques est void *. Comme ces pointeurs sont génériques, la taille des données pointées est inconnue et l'arithmétique des pointeurs ne s'applique donc pas à eux. De même, puisque la taille des données pointées est inconnue, l'opérateur d'indirection * ne peut être utilisé avec ces pointeurs, un cast est alors obligatoire. Par exemple:
\begin{lstlisting}
  int n;
  void * p;

  p = &n;
  *((int *)p) = 10;
  /* p étant désormais vu comme un int *, on peut alors lui appliquer l'opérateur *.
  */
\end{lstlisting}
Étant donné que la taille de toute donnée est multiple de celle d'un char, le type char * peut également être utilisé en tant que pointeur universel. En effet, une variable de type char * est un pointeur sur octet autrement dit il peut pointer n'importe quoi. Cela s'avère pratique de temps en temps (lorsqu'on veut lire le contenu d'une mémoire octet par octet par exemple) mais dans la plupart des cas, il vaut mieux toujours utiliser les pointeurs génériques. Par exemple, la conversion d'une adresse de type différent en char * et vice versa nécessite toujours un cast, ce qui n'est pas le cas avec les pointeurs génériques.\newline
Dans printf, le spécificateur de format \%p permet d'imprimer une adresse (void *) dans le format utilisé par le système.\newline
Et pour terminer, il existe une macro à savoir NULL, définie dans stddef.h, permettant d'indiquer qu'un pointeur ne pointe nulle part. Son intérêt est donc de permettre de tester la validité d'un pointeur.Il est conseillé de toujours initialiser un pointeur à NULL.

\subsection{lvalue et rvalue}
Dois-je vraiment rappeler comment ça fonctionne?
En gros, une lvalue est quelque chose qui peut ce situer à gauche du \guillemotleft{=}\guillemotright{} et rvalue a droite. Une rvalue possède une valeur mais pas d'adresse.
\href{https://www.internalpointers.com/post/understanding-meaning-lvalues-and-rvalues-c}{Très bon article (en anglais) sur les lvalues et rvalues}
\subsection{Opérateurs usuels et logiques}
\textbf{usuels}: %mis en gras
\begin{description}
  \item [<] 	Inférieur à
  \item [>] 	Supérieur à
  \item [==] 	Égal à
  \item [<=] 	Inférieur ou égal à
  \item [>=] 	Supérieur ou égal à
  \item [!=] 	Différent de
\end{description}
\textbf{logiques}:
\begin{description}
  \item [\&\&] ET
  \item [||] OU
  \item [!] NON
\end{description}
Dans une opération ET, l'évaluation se fait de gauche à droite. Si l'expression à gauche de l'opérateur est fausse, l'expression à droite ne sera plus évaluée car on sait déjà que le résultat de l'opération sera toujours FAUX.\newline
Dans une opération OU, l'évaluation se fait de gauche à droite. Si l'expression à gauche de l'opérateur est vrai, l'expression à droite ne sera plus évaluée car on sait déjà que le résultat de l'opération sera toujours VRAI.


On peut séparer plusieurs expressions à l'aide de l'opérateur virgule. Le résultat est une expression dont la valeur est celle de l'expression la plus à droite. L'expression est évaluée de gauche à droite.
\begin{lstlisting}
  (a = -5, b = 12, c = a + b) * 2 ;   // renvoie 14.
\end{lstlisting}
\subsubsection{l'opérateur ternaire \guillemotleft{}? : \guillemotright{}}
Une expression conditionnelle est une expression dont la valeur dépend d'une condition. L'expression:
\begin{lstlisting}
  p ? a : b ;
\end{lstlisting}
vaut a si p est vrai et b si p est faux.
\subsection{Cast}
\begin{lstlisting}
float f;
f = (float)3.1416;
\end{lstlisting}
\section{Les instructions}
\subsection{if}
\begin{lstlisting}
if ( <expression> )
  {
    les instructions
  }
else if (<expression>)
{
  les instructions
}
else
  {
    les instructions
  }
\end{lstlisting}
\subsection{do ... while}
\begin{lstlisting}
do
  {
    les instructions
  }
while ( <expression> );
\end{lstlisting}
\subsection{while}
\begin{lstlisting}
while ( <expression> )
  {
    les instructions
  }
\end{lstlisting}
\subsection{for}
\begin{lstlisting}
for ( <init> ; <condition> ; <step>)
  les instructions
\end{lstlisting}
\subsection{break}
%https://msdn.microsoft.com/fr-fr/library/wt88dxx6.aspx
L'instruction break termine l'exécution de l'instruction $do$, $for$, $switch$ ou $while$ englobante la plus proche dans laquelle elle figure. Le contrôle est transmis à l'instruction qui suit l'instruction terminée.\subsection{switch}
\begin{lstlisting}
#include <stdio.h>

int main()
{
  int n;

  printf("Entrez un nombre entier : ");
  scanf("%d", &n);

  switch(n)
  {
  case 0:
    printf("Cas de 0.\n");
    break;
  case 1:
    printf("Cas de 1.\n");
    break;
  case 2: case 3:
    printf("Cas de 2 ou 3.\n");
    break;
  case 4:
    printf("Cas de 4.\n");
    break;
  default:
    printf("Cas inconnu.\n");
  }
  return 0;
}
\end{lstlisting}
\subsection{continue}
Dans une boucle, permet de passer immédiatement à l'itération suivante. Par exemple, modifions le programme table de multiplication de telle sorte qu'on affiche rien pour n = 4 ou n = 6.
\begin{lstlisting}
  #include <stdio.h>

  int main()
  {
    int n;
    for(n = 0; n <= 10; n++)
    {
      if ((n == 4) || (n == 6))
      continue; // termine la boucle actuelle du for
    printf("5 x %2d %2d\n", n, 5 * n);
    }
    return 0;
  }
\end{lstlisting}
\section{L'allocation dynamique de mémoire}
\subsection{Les fonctions malloc et free}
L'intérêt d'allouer dynamiquement de la mémoire se ressent lorsqu'on veut créer un tableau dont la taille dont nous avons besoin n'est connue qu'à l'exécution par exemple. On utilise généralement les fonctions malloc et free.
\begin{lstlisting}
  int t[10];
  ...
  /* FIN */
\end{lstlisting}
Peut être remplacé par:
\begin{lstlisting}
  int * p;

  p = malloc(10 * sizeof(int));
  ...
  free(p); /* libérer la mémoire lorsqu'on n'en a plus besoin */
  /* FIN */
\end{lstlisting}
Les fonctions malloc et free sont déclarées dans le fichier stdlib.h. malloc retourne NULL en cas d'échec. Voici un exemple qui illustre une bonne manière de les utiliser:
\begin{lstlisting}
  #include <stdio.h>
  #include <stdlib.h>

  int main()
  {
    int * p;

    /* Creation d'un tableau assez grand pour contenir 10 entiers */
    p = malloc(10 * sizeof(int));

    if (p != NULL)
    {
      printf("Succes de l'operation.\n");
      p[0] = 1;
      printf("p[0] = %d\n", p[0]);
      free(p); /* Destruction du tableau. */
    }
    else
      printf("Le tableau n'a pas pu etre cree.\n");
    return 0;
  }
\end{lstlisting}

\subsection{La fonction realloc}
\begin{lstlisting}
void * realloc(void * memblock, size_t newsize);
\end{lstlisting}
Permet de « redimensionner » une mémoire allouée dynamiquement (par malloc par exemple). Si memblock vaut NULL, realloc se comporte comme malloc. En cas de réussite, cette fonction retourne alors l'adresse de la nouvelle mémoire, sinon la valeur NULL est retournée et la mémoire pointée par memblock reste inchangée.
\begin{lstlisting}
#include <stdio.h>
#include <stdlib.h>

int main() {
    int * p = malloc(10 * sizeof(int));

    if (p != NULL) {
        /* Sauver l'ancienne valeur de p au cas ou realloc echoue. */
        int * q = p;
        /* Redimensionner le tableau. */
        p = realloc(p, 20 * sizeof(int));

        if (p != NULL) {
            printf("Succes de l'operation.\n");
            p[0] = 1;
            printf("p[0] = %d\n", p[0]);
            free(p);
        }
        else {
            printf("Le tableau n'a pas pu etre redimensionne.\n");
            free(q);
        }
    }
    else
        printf("Le tableau n'a pas pu etre cree.\n");

    return 0;
}
\end{lstlisting}
\subsection{Les structs}
Les structures permettent de remédier à cette lacune des tableaux, en regroupant des objets (des variables) de types différentes au sein d'une entité repérée par un seul nom de variable. \newline
Lors de la déclaration de la structure, on indique les champs de la structure, c'est-à-dire le type et le nom des variables qui la composent.
\begin{lstlisting}
struct Nom_Structure {
  type_champ1 Nom_Champ1;
  type_champ2 Nom_Champ2;
  type_champ3 Nom_Champ3;
  type_champ4 Nom_Champ4;
  type_champ5 Nom_Champ5;
  ...
};
// son utilisation
struct Nom_Structure Nom_Variable_Structuree;
\end{lstlisting}
Exemple:
\begin{lstlisting}
struct MaStructure {
  int Age;
  char Sexe;
  char Nom[12];
  float MoyenneScolaire;
  struct AutreStructure StructBis;
/* en considérant que la structure AutreStructure est définie */
};
//
struct MaStructure Pierre, Paul;
Pierre.Age=42;
\end{lstlisting}
\section{Les caractères}
\subsection{Le jeu de caractères du C}
Voir \href{http://emmanuel-delahaye.developpez.com/tutoriels/c/bonnes-pratiques-codage-c/#LI-A}{les bonnes pratiques de codage en C}
\subsection{Commentaire}
Préférez les commentaires en dessous ou au dessus d'une ligne plutôt qu'en bout de ligne, souvent ils rendent la ligne trop longue. \newline
Pour les commentaires de plusieurs lignes, utiliser /* */.\newline
S'il faut isoler provisoirement une portion de code, le mieux est de ne pas utiliser les commentaires (il pourrait y avoir des commentaires imbriqués), mais plutôt les directives de préprocesseur: \#ifdef .. \#endif ou \#if .. \#endif

\begin{lstlisting}
#if 0
  /* Compteur */
  int cpt ;
#endif
\end{lstlisting}
Commentez le moins possible. Le principe est de ne commenter que ce qui apporte un supplément d'information. Il est d'usage d'utiliser en priorité l'auto-documentation, c'est à dire un choix judicieux des identificateurs qui fait que le code se lit 'comme un livre'…
\section{Les threads}
On sait tous ce qu'est un thread, si tu ne sais pas, tu es dans la merde pour ton oral et va d'abord (re)voir ton cours!
\subsection{Compilation}
Toutes les fonctions relatives aux threads sont incluses dans le fichier d'en-tête <pthread.h> et dans la bibliothèque libpthread.a (soit -lpthread à la compilation).\newline
Exemple:\newline
Voici la ligne de commande qui vous permet de compiler votre programme sur les threads constitué d'un seul fichier.\newline
gcc -lpthread <nom du fichier>.c -o <Output>\newline
Et n'oubliez pas d'ajouter \#include <pthread.h> au début de vos fichiers.
Bon, tout ça on aura pas à l'examen, c'est juste pour le projet de C.
\subsection{Les threads en eux-mêmes}
\subsubsection{Créer un thread}
Pour créer un thread, il faut déjà déclarer une variable le représentant.Celle-ci sera de type pthread\textunderscore{}t (qui est, sur la plupart des systèmes, un typedef d'unsigned long int). Ensuite, pour créer la tâche elle-même, il suffit d'utiliser la fonction:
\begin{lstlisting}
#include <pthread.h>

  int pthread_create(pthread_t * thread, pthread_attr_t * attr,
                     void *(*start_routine) (void *), void *arg);
\end{lstlisting}
Ce prototype est un peu compliqué, c'est pourquoi nous allons récapituler ensemble.
\begin{description}
  \item La fonction renvoie une valeur de type int: 0 si la création a été réussie ou une autre valeur s'il y a eu une erreur.
  \item Le premier argument est un pointeur vers l'identifiant du thread (valeur de type pthread\textunderscore{}t).
  \item Le second argument désigne les attributs du thread. Vous pouvez choisir de mettre le thread en état joignable (par défaut) ou détaché, et choisir sa politique d'ordonnancement (usuelle, temps-réel\ldots). Dans nos exemples, on mettra généralement NULL.
  \item Le troisième argument est un pointeur vers la fonction à exécuter dans le thread. Cette dernière devra être de la forme \guillemotleft{}void *fonction(void* arg)\guillemotright{} et contiendra le code à exécuter par le thread.
  \item Enfin, le quatrième et dernier argument est l'argument à passer au thread.
\end{description}
\subsubsection{Supprimer un thread}
\begin{lstlisting}
#include <pthread.h>

void pthread_exit(void *ret);
\end{lstlisting}
Elle prend en argument la valeur qui doit être retournée par le thread, et doit être placée en dernière position dans la fonction concernée.
\newpage
\subsubsection{Exemple simple}
\begin{lstlisting}
#include <stdio.h>
#include <stdlib.h>
#include <unistd.h>
#include <pthread.h>

void *thread_1(void *arg) {
    printf("Nous sommes dans le thread.\n");

    /* Pour enlever le warning */
    (void) arg;  // le warning enlevé est celui-ci:
    /* format '%s' expects argument of type 'char*',
    *  but argument 2 has type 'void*' [-Wformat=]
    *  en gros, il faut reconvertir l'argument
    *  s'il y en a un pour pas avoir d'erreur de format */
    pthread_exit(NULL);
}

int main(void) {
    pthread_t thread1;  // thread1 contiendra le thread

    printf("Avant la création du thread.\n");

    /* appelle la fonction thread_1 dans la condition du if
    * (pour direct renvoyer une erreur si besoin) */
    if(pthread_create(&thread1, NULL, thread_1, NULL) == -1) {
	    perror("pthread_create");
      return EXIT_FAILURE;
    }
    printf("Après la création du thread.\n");

    return EXIT_SUCCESS;
}
\end{lstlisting}
On a bien sur un soucis, c'est que ici, le programme s'arrête avant que le thread ait pu finir son calcul (bon, ici, ce n'est qu'afficher quelque chose, mais quand-même). Pour éviter ce soucis, on peut utiliser \guillemotleft{} pthread\textunderscore{}join \guillemotright{}
\subsubsection{Attendre un thread avec pthread\textunderscore{}join}
\begin{lstlisting}
#include <pthread.h>

int pthread_join(pthread_t th, void **thread_return);
\end{lstlisting}
Elle prend donc en paramètre l'identifiant du thread et son second paramètre, un pointeur, permet de récupérer la valeur retournée par la fonction dans laquelle s'exécute le thread (c'est-à-dire l'argument de pthread\textunderscore{}exit).\newline
Du coup dans l'exemple précédent, on peut rajouter ceci(après la création du thread, bien sûr):
\begin{lstlisting}
  if (pthread_join(thread1, NULL)) {
    perror("pthread_join");
    return EXIT_FAILURE;
    }
\end{lstlisting}
Donc, pour ceux qui n'ont pas suivi:
\begin{enumerate}
  \item créer une fonction qui va permettre de faire tous les calculs qu'on veut dans les autres threads que le thread principal;
  \item initialiser une variable de type pthread\textunderscore{}t pour chaque thread qu'on veut créer;
  \item appeler cette fonction (dans un if, souvent, ça permet de gérer les éventuelles erreurs plus facilement);
  \item appeler la fonction pthread\textunderscore{}join pour "fusionner" les deux threads (en réalité, attendre le thread en retard sur l'autre)
\end{enumerate}
\subsection{Mutex et toute ces belles petites choses}
Bon, c'est sympa tout ça, mais il y a encore un soucis. Toutes les variables sont partagées, et on se rend vite compte que ça peut poser problème, si on veut faire plusieurs calculs sur une même variable, mais l'un à la suite de l'autre, ça risque de coincer, on ne sait pas dans quel ordre vont se faire les calculs. Ce qui veut dire que nous risquons de modifier une valeur dans un thread alors qu'un autre thread en avait besoin sans modifications.(je ne sais pas si j'en ai pas perdu par hasard, si c'est le cas, sachez juste que les mutexes sont vachement utiles pour garder visible une variable qu'à un seul thread et la cacher à tous les autres).\newline
Il faut aussi de nouveau créer une variable, mais cette fois avec un type \guillemotleft{}pthread\textunderscore{}mutex\textunderscore{}t\guillemotright{}.\newline
Le problème, c'est qu'il faut que le mutex soit accessible en même temps que la variable et dans tout le fichier (vu que différents threads s'exécutent dans différentes fonctions). La solution la plus simple consiste à déclarer les mutexes en variable globale. Pour le faire plus proprement possible, on peut utiliser une structure avec la donnée à protéger. Allez, encore un exemple:
\begin{lstlisting}
typedef struct data {
    int var;
    pthread_mutex_t mutex;
} data;
\end{lstlisting}
Ainsi, nous pourrons passer la structure en paramètre à nos threads quand nous les créons.\newline
Du coup, voilà comment on initialise un mutex en prenant en compte la convention qui veut qu'on initialise le mutex avec la valeur de la constante PTHREAD\textunderscore{}MUTEX\textunderscore{}INITIALIZER, déclarée dans pthread.h.
\begin{lstlisting}
#include <stdlib.h>
#include <pthread.h>

typedef struct data {
    int var;
    pthread_mutex_t mutex;
} data;

int main(void){
    data new_data;

    new_data.mutex = PTHREAD_MUTEX_INITIALIZER;

    return EXIT_SUCCESS;
}
\end{lstlisting}

\subsubsection{Pour verrouiller un mutex de nom mut}
\begin{lstlisting}
int pthread_mutex_lock(pthread_mutex_t *mut);
\end{lstlisting}


\subsubsection{Pour déverrouiller un mutex de nom mut}
\begin{lstlisting}
int pthread_mutex_unlock(pthread_mutex_t *mut);
\end{lstlisting}
\subsubsection{Pour détruire un mutex de nom mut}
\begin{lstlisting}
#include <pthread.h>
int pthread_mutex_destroy(pthread_mutex_t *mut);
\end{lstlisting}
\subsection{Les conditions (permettent d'attendre un autre thread)}
Quand un thread est en attente d'une condition, il reste bloqué tant que celle-ci n'est pas réalisée par un autre thread.
\subsubsection{Initialisation} %tu voulais pas dire initialisation ? si si complètement
\begin{lstlisting}
  static pthread_cond_t cond_stock = PTHREAD_COND_INITIALIZER;
\end{lstlisting}
Les conditions reposent essentiellement sur deux fonctions.
l'une permet de mettre en attente un thread et la seconde permet de signaler que la condition est remplie ce qui réveille alors le thread qui est en attente de cette condition.
Plusieurs threads peuvent surveiller la même condition.
\newpage
\subsubsection{Exemple complet de mutex}
\begin{lstlisting}
#include<stdio.h>
#include<string.h>
#include<pthread.h>
#include<stdlib.h>
#include<unistd.h>

pthread_t tid[2];
int counter;
pthread_mutex_t lock;  // init le mutex

void* doSomeThing(void *arg){
    pthread_mutex_lock(&lock);

    unsigned long i = 0;
    counter += 1;

    printf("\n Job %d started\n", counter);
    for(i=0; i<(0xFFFFFFFF);i++);
    printf("\n Job %d finished\n", counter);

    pthread_mutex_unlock(&lock);

    return NULL;
}

int main(void){
    int i = 0;
    int err;

    if (pthread_mutex_init(&lock, NULL) != 0){
        printf("\n mutex init failed\n");
        return 1;
    }
    while(i < 2){
        err = pthread_create(&(tid[i]), NULL, &doSomeThing, NULL);
        if (err != 0)
            printf("\ncan't create thread :[%s]", strerror(err));
        i++;
    }
    pthread_join(tid[0], NULL);
    pthread_join(tid[1], NULL);
    pthread_mutex_destroy(&lock);
    return 0;
}
\end{lstlisting}
\href{https://www.thegeekstuff.com/2012/05/c-mutex-examples/}{ma source pour cet exemple}
\newpage
\subsubsection{Le thread attend la condition}
\begin{lstlisting}
int pthread_cond_wait (pthread_cond_t *cond, pthread_mutex_t *mutex);
\end{lstlisting}
Cette fonction permet de mettre le thread appelant en attente de la condition, il suspend donc son exécution temporairement.
Ses deux arguments sont:
\begin{description}
  \item [L'adresse de la variable] condition de type pthread\textunderscore{}cond\textunderscore{}t.
  \item [L'adresse d'un mutex] Une condition est en effet, toujours associée à un mutex!
\end{description}
\subsubsection{Réveiller un thread}
pthread\textunderscore{}cond\textunderscore{}signal est la fonction qui permet de signaler la condition au thread qui l'attend.
Elle prend en paramètre l'adresse de la variable-condition surveillée. Cette fonction ne permet de réveiller qu'un seul thread.
\begin{lstlisting}
int pthread_cond_signal (pthread_cond_t *cond);
\end{lstlisting}
\subsubsection{Réveiller plusieurs threads}
Cette fonction permet de réveiller tous les threads qui surveillent la condition cond. Tout comme\newline pthread\textunderscore{}cond\textunderscore{}signal, elle prend en paramètre l'adresse de la variable-condition surveillée.
\begin{lstlisting}
int pthread_cond_broadcast (pthread_cond_t *cond);
\end{lstlisting}
\href{http://franckh.developpez.com/tutoriels/posix/pthreads/}{cours complet sur les threads}
\newpage
\subsection{Exemple avec les conditions et les mutexes}
\begin{lstlisting}
#include <stdio.h>
#include <stdlib.h>
#include <pthread.h>

// Création de la condition
pthread_cond_t condition = PTHREAD_COND_INITIALIZER;
//Création du mutex
pthread_mutex_t mutex = PTHREAD_MUTEX_INITIALIZER;

void* threadAlarme (void* arg);
void* threadCompteur (void* arg);

int main (void){
	pthread_t monThreadCompteur;
	pthread_t monThreadAlarme;

	pthread_create (&monThreadCompteur, NULL, threadCompteur, (void*)NULL);
    // Création des threads
	pthread_create (&monThreadAlarme, NULL, threadAlarme, (void*)NULL);

	pthread_join (monThreadCompteur, NULL);
    // Attente de la fin des threads
	pthread_join (monThreadAlarme, NULL);

	return 0;
}

void* threadCompteur (void* arg) {
	int compteur = 0, nombre = 0;

	srand(time(NULL));

    // Boucle infinie
	while(1) {
        // On tire un nombre entre 0 et 10
		nombre = rand()%10;
        // On ajoute ce nombre à la variable compteur
		compteur += nombre;

		printf("\n%d", compteur);

        // Si compteur est plus grand ou égal à 20
		if(compteur >= 20) {
            // On verrouille le mutex
			pthread_mutex_lock (&mutex);
            // On délivre le signal : condition remplie
			pthread_cond_signal (&condition);
            // On déverrouille le mutex
			pthread_mutex_unlock (&mutex);

            // On remet la variable compteur à 0
			compteur = 0;
		}
        // On laisse 1 seconde de repos
		sleep (1);
	}
    // Fin du thread
	pthread_exit(NULL);
}

void* threadAlarme (void* arg) {
    // Boucle infinie
	while(1) {
        // On verrouille le mutex
		pthread_mutex_lock(&mutex);
        // On attend que la condition soit remplie
		pthread_cond_wait (&condition, &mutex);
		printf("\nLE COMPTEUR A DÉPASSÉ 20.");
        // On déverrouille le mutex
		pthread_mutex_unlock(&mutex);
	}
    // Fin du thread
	pthread_exit(NULL);
}
\end{lstlisting}
\href{https://openclassrooms.com/courses/la-programmation-systeme-en-c-sous-unix/les-threads-3#/id/r-1515355}{Lien de l'exemple}
\section{Gestion des processus}
Pour cette partie je me suis plus inspiré des slides des tps.
\subsection{Création des processus}
\subsubsection{fork}
Fork permet de cloner intégralement le processus courant avec un pid différent pour son fils.
Du coup, fork est la seule fonction qui renvoie deux valeurs:
\begin{enumerate}
  \item dans le processus fils (le clone), fork renvoie 0
  \item dans le processus père (l'original), fork renvoie le pid du fils
\end{enumerate}
Exemple:
\begin{lstlisting}
#include <sys/types.h>
#include <unistd.h>

pid_t fork();
// renvoie 0 si c'est l'enfant
// < 0 si il y a une erreur dans le fork
// > 0 si c'est le parent
\end{lstlisting}
\newpage
\textbf{Exemple complet:}
\begin{lstlisting}
#include <unistd.h>
#include <sys/types.h>
#include <errno.h>
#include <stdio.h>
#include <sys/wait.h>
#include <stdlib.h>

int var_glb; /* A global variable*/

int main(void) {
  pid_t childPID;
  int var_lcl = 0;

  childPID = fork();

  if(childPID >= 0) { // fork was successful
    if(childPID == 0) { // child process
      var_lcl++;
      var_glb++;
      printf("\n Child :: var_lcl = [%d], var_glb[%d]\n", var_lcl, var_glb);
    }
    else { //Parent process
      var_lcl = 10;
      var_glb = 20;
      printf("\n Parent :: var_lcl = [%d], var_glb[%d]\n", var_lcl, var_glb);
    }
  }
  else { // fork failed
    printf("\n Fork failed, quitting!!!!!!\n");
    return 1;
  }
  return 0;
}
\end{lstlisting}
Les variables var\textunderscore{}lcl et var\textunderscore{}glb sont utilisées pour montrer que le père et le fils travaillent bien sur des variables séparées.\newline
Output:\newline
Parent :: var\textunderscore{}lcl = [10], var\textunderscore{}glb[20]\newline
Child :: var\textunderscore{}lcl = [1], var\textunderscore{}glb[1]\newline
\href{https://www.thegeekstuff.com/2012/05/c-fork-function/}{Liens de l'exemple}\newline
\href{http://www.commentcamarche.net/faq/10611-que-fait-un-fork}{Fork pas à pas}\newline
\href{http://pubs.opengroup.org/onlinepubs/9699919799/functions/fork.html#}{Référence anglais pour fork}\newline
\subsubsection{getpid()}
Comment savoir le pid du processus courant?
\begin{lstlisting}
  #include <sys/types.h>
  #include <unistd.h>
  pid_t getpid(void);
\end{lstlisting}
Et pour le pid du père:
\begin{lstlisting}
  #include <sys/types.h>
  #include <unistd.h>
  pid_t getppid(void);
\end{lstlisting}
\subsection{Exécution d'un programme depuis un programme C, "system calls"}
Utilise la famille exec\newline
Lorsqu'un processus fait appel à exec, le processus appelant est remplacé par le programme passé en paramètre. Si cela fonctionne, l'appel à exec ne retourne pas.\newline
les différentes variantes:
\begin{lstlisting}
// permet de passer les paramètres du programme
// en utilisant une liste d'arguments terminée par NULL
int execl(const char *program, const char *arg, ...);
execl("/bin/ls", "/bin/ls", "-l", "-a", NULL );

// cherche le programme dans la variable PATH
int execlp(const char *program, const char *arg, ...);
execlp("ls", "ls", "-l", "-a", NULL );

// avec un vecteur pour les paramètres et dois donner le chemin absolu
int execv(const char *program, const char *argv [ ]);
char *arguments [4];
arguments[0] = "/bin/ls";
arguments[1] = "-l";
arguments[2] = "-a";
arguments[3] = NULL;
execv("/bin/ls", arguments);


// avec un vecteur pour les paramètres et dois donner le chemin absolu
int execvp(const char *program, const char *argv [ ]);
char *arguments [4];
arguments[0] = "ls";
arguments[1] = "-l";
arguments[2] = "-a";
arguments[3] = NULL;
execvp("ls", arguments);
\end{lstlisting}
\href{http://pubs.opengroup.org/onlinepubs/9699919799/functions/exec.html}{Référence en anglais pour approfondir}
\subsubsection{Attendre la fin d'un processus}
Attendre la fin d'un processus peut être intéressant quand on attend de ce processus un résultat.\newline
Utilise la famille exec\newline
\begin{lstlisting}
#include<sys/types.h>
#include<sys/wait.h>

pid_t wait(int *status);
pid_t waitpid(pid_t pid, int *status, int options);
\end{lstlisting}
wait: attend la fin d'un des processus fils, retourne son PID et stocke dans l'entier pointé par status (si status est différent de NULL) le code de retour du processus fils (ce qui est renvoyé par la fonction main du processus fils).\par{}
waitpid: comme wait mais attend uniquement le processus fils précisé par le paramètre PID.  Le paramètre entier \guillemotleft{}option\guillemotright{} permet de rendre l'appel à waitpid non bloquant avec la constante WNOHANG, si cela n'est pas nécessaire, on utilise la valeur 0.
\par{}
Pour savoir si le processus s'est terminé correctement, on utile la macro WIFEXITED(status) qui renvoie vrai si le processus fils s'est terminé normalement.\newline

Ensuite pour avoir le résultat que le processus a renvoyé (ce que son main a renvoyé), on utilise la macro WEXITSTATUS(status).\newpage
Et voilà, encore bien sûr un exemple:
\begin{lstlisting}
/* CELEBW02

  The following function suspends the calling process using &waitpid.
  until a child process ends.

*/
#define _POSIX_SOURCE
#include <sys/types.h>
#include <sys/wait.h>
#include <unistd.h>
#include <stdio.h>
#include <time.h>

main() {
  pid_t pid;
  time_t t;
  int status;

  if ((pid = fork()) < 0)
    perror("fork() error");
  else if (pid == 0) {
    sleep(5);
    exit(1);
  }
  else do {
    if ((pid = waitpid(pid, &status, WNOHANG)) == -1)
      perror("wait() error");
    else if (pid == 0) {
      time(&t);
      printf("child is still running at %s", ctime(&t));
      sleep(1);
    }
    else {
      if (WIFEXITED(status))
        printf("child exited with status of %d\n", WEXITSTATUS(status));
      else puts("child did not exit successfully");
    }
  }
  while (pid == 0);
}
\end{lstlisting}
Output:\newline
child is still running at Mon Jan 8 11:05:43 2018\newline
child is still running at Mon Jan 8 11:05:44 2018\newline
child is still running at Mon Jan 8 11:05:45 2018\newline
child is still running at Mon Jan 8 11:05:46 2018\newline
child is still running at Mon Jan 8 11:05:47 2018\newline
child is still running at Mon Jan 8 11:05:48 2018\newline
child is still running at Mon Jan 8 11:05:49 2018\newline
child exited with status of 1\newline
\href{https://www.ibm.com/support/knowledgecenter/en/SSLTBW_2.1.0/com.ibm.zos.v2r1.bpxbd00/rtwaip.htm}{Liens de l'exemple avec des explications plus poussées (anglais)}
\subsection{Les pipes}
Les pipes permettent à deux processus sur la même machine de communiquer de manière unidirectionnelles entre eux. (pour pouvoir communiquer entre des machines différentes, on va utiliser des serveurs, mais on verra ça plus tard).\newline
Il en existe deux types:
\begin{enumerate}
  \item anonymes: communication parentale, en famille quoi (père -> fils; fils -> fils)
  \item nommés: utilisés pour la communication entre deux processus indépendants (pas de lien de parenté direct entre les deux), du coup il faudra le nommer, d'où le nom.
\end{enumerate}
\subsubsection{Comment les créer?}
\textbf{Les pipes anonymes:}
\begin{lstlisting}
  #include <unistd.h>

  int pipe(int fd[2]);
\end{lstlisting}
C'est quoi tout ça?\newline
\begin{description}
  \item L'appel à pipe crée une paire de \guillemotleft{}file descriptors\guillemotright{} stockée dans fd.\newline
           fd[0] sert à la lecture du pipe et fd[1] sert à l'écriture dans le pipe.
  \item La fonction pipe renvoie 0 si elle réussit, autre chose sinon.
  \item Pour établir une communication (unidirectionelle!) à l'aide d'un pipe, il faut que le processus père crée un pipe avant de faire un (ou plusieurs) appel(s) à fork (sinon l'autre processus ne pourra pas connaître l'entrée ou la sortie du pipe). Dès lors, après le fork, le processus père et le(s) processus fils peuvent accéder au pipe en utilisant fd.
  \item Pour une communication bidirectionelle, il faut bien sûr créer deux pipes.
\end{description}
Pour écrire dans un tube (ce n'est pas dans les tps, mais ça me semble plutôt utile):
\begin{lstlisting}
ssize_t write(int entreeTube, const void *elementAEcrire, size_t nombreOctetsAEcrire);
\end{lstlisting}
La fonction prend en paramètre l'entrée du tube (on lui enverra fd[1]), un pointeur générique vers la mémoire contenant l'élément à écrire, ainsi que le nombre d'octets de cet élément.\newline
Elle renvoie une valeur de type ssize\textunderscore{}t correspondant au nombre d'octets effectivement écrits.\par
Et pour lire dedans:
\begin{lstlisting}
ssize_t read(int sortieTube, void *elementALire, size_t nombreOctetsALire);
\end{lstlisting}
La fonction prend en paramètre la sortie du tube (fd[0]), un pointeur vers la mémoire contenant l'élément à lire et le nombre d'octets de cet élément.\newline
Elle renvoie une valeur de type ssize\textunderscore{}t qui correspond au nombre d'octets effectivement lus. On pourra ainsi comparer le troisième paramètre (nombreOctetsALire) à la valeur renvoyée pour vérifier qu'il n'y a pas eu d'erreurs.\newpage
Petit exemple offert par OpenClassrooms:
\begin{lstlisting}
#include <stdio.h>
#include <stdlib.h>
#include <unistd.h>
#include <sys/wait.h>

#define TAILLE_MESSAGE 256 /* Correspond à la taille de la chaîne à écrire */

int main(void) {
    pid_t pid_fils;
    int fd[2];

    char messageEcrire[TAILLE_MESSAGE];

    if (pipe(fd) != 0) {
      perror "erreur a la créeation du pipe"
    }
    pid_fils = fork();

    if(pid_fils != 0) { /* Processus père */
        sprintf(messageEcrire, "Bonjour, fils. Je suis ton père !");
        /* La fonction sprintf permet de remplir
        *  une chaîne de caractère avec un texte donné */

        write(fd[1], messageEcrire, TAILLE_MESSAGE);
    }
    else if(pid_fils == 0) /* Processus fils */
    {
        read(fd[0], messageLire, TAILLE_MESSAGE);
        printf("Message reçu = \"%s\"", messageLire);
    }
    return EXIT_SUCCESS;
}
\end{lstlisting}
\textbf{Les pipes nommés:}
\begin{lstlisting}
  #include <sys/types.h>
  #include <sys/stat.h>

  int mkfifo(const char *pathname, mode_t mode);
\end{lstlisting}
Le pipe nommé passe par un fichier spécial qui est créer lors de l'appel de cette fonction,
\begin{description}
  \item avec comme path le pathname donné en paramètre
  \item et comme permissions, ce qui est indiqué dans le paramètre mode, il y a plusieurs moyen d'utiliser ce paramètre:
  \begin{enumerate}
    \item S\textunderscore{}IRWXU: read, write, execute/search by owner;
    \item S\textunderscore{}IRUSR: read permission, owner;
    \item S\textunderscore{}IWUSR: write permission, owner;
    \item S\textunderscore{}IXUSR: execute/search permission, owner;
    \item S\textunderscore{}IRWXG: read, write, execute/search by group;
    \item S\textunderscore{}IRGRP: read permission, group;
    \item S\textunderscore{}IWGRP: write permission, group;
    \item S\textunderscore{}IXGRP: execute/search permission, group;
    \item S\textunderscore{}IRWXO: read, write, execute/search by others;
    \item S\textunderscore{}IROTH: read permission, others;
    \item S\textunderscore{}IWOTH: write permission, others;
    \item S\textunderscore{}IXOTH: execute/search permission, others;
    \item S\textunderscore{}ISUID: set-user-ID on execution;
    \item S\textunderscore{}ISGID: set-group-ID on execution;
    \item S\textunderscore{}ISVTX: on directories, restricted deletion flag;
    \item Ils peuvent être combiné avec: \guillemotleft{}|\guillemotright{}
    \item Ou encore utiliser les valeurs des droits avec des chiffres(sous forme octal):
    \begin{enumerate}
      \item Le premier chiffre correspond au propriétaire
      \item Le second au groupe du propriétaire
      \item Le troisième à tous les autres \newline
      \item Chaque chiffre peut avoir une valeur différente:
      \begin{enumerate}
        \item 1 exécution
        \item 2 écriture
        \item 4 lecture\newline
      \end{enumerate}
      Et on fait une somme quand on veut combiner plusieurs permissions\par
      Exemple:\newline
      Pour attribuer toutes les permissions à vous, seule la lecture pour le groupe, et aucune pour les autres, la valeur correspondante est 0720\newline
      Premier chiffre = 0 (obligatoire)\newline
      Deuxième chiffre = 1 (Exécution) + 2 (Ecriture) + 4 (Lecture)\newline
      Troisième chiffre = 0 (aucune permission)\newline
    \end{enumerate}
  \end{enumerate}
  \item Et renvoie 0 si elle réussit, ou -1 en cas d'erreur.
\end{description}
\newpage
Un dernier "petit" exemple:  (Ecrivain.c)
\begin{lstlisting}
#include <fcntl.h>
#include <stdio.h>
#include <stdlib.h>
#include <unistd.h>

#define TAILLE_MESSAGE	256

int main(void) {
	int entreeTube;
	char nomTube[] = "essai.fifo";
	char chaineAEcrire[TAILLE_MESSAGE] = "Bonjour";

	if(mkfifo(nomTube, 0644) != 0) 	{ // créer le tube
		fprintf(stderr, "Impossible de créer le tube nommé.\n");
		exit(EXIT_FAILURE);
	}

	if((entreeTube = open(nomTube, O_WRONLY)) == -1) 	{
		fprintf(stderr, "Impossible d'ouvrir l'entrée du tube nommé.\n");
		exit(EXIT_FAILURE);
	}

	write(entreeTube, chaineAEcrire, TAILLE_MESSAGE); // écris dans le tube

	return EXIT_SUCCESS;
}
\end{lstlisting}
Lecteur.c
\begin{lstlisting}
#include <fcntl.h>
#include <stdio.h>
#include <stdlib.h>
#include <unistd.h>

#define TAILLE_MESSAGE	256

int main(void) {
	int sortieTube;
	char nomTube[] = "essai.fifo";
	char chaineALire[TAILLE_MESSAGE];

	if((sortieTube = open ("essai.fifo", O_RDONLY)) == -1) 	{
		fprintf(stderr, "Impossible d'ouvrir la sortie du tube nommé.\n");
		exit(EXIT_FAILURE);
	}

	read(sortieTube, chaineALire, TAILLE_MESSAGE); // lit dans le tube
        printf("%s", chaineALire);

	return EXIT_SUCCESS;
}

\end{lstlisting}
\href{https://openclassrooms.com/courses/la-programmation-systeme-en-c-sous-unix/les-tubes}{Les pipes (deux exemples), OpenClassrooms}\newline
\href{http://www.zeitoun.net/articles/communication-par-tuyau/start}{Excellent site pour les pipes}
\newpage
\section{Les serveurs}
Les fonctions vues et utiles:
\begin{enumerate}
  \item socket
  \item bind
  \item connect
  \item listen
  \item accept
  \item send
  \item write?
  \item recv
  \item shutdown
  \item getpeername
  \item gethostname
  \item gethostbyname
  \item select
\end{enumerate}
\href{http://broux.developpez.com/articles/c/sockets/}{Cours sur developpez.com}
Les sockets sont des flux de données, permettant à des machines locales ou distantes de communiquer entre elles via des protocoles. Les différents protocoles sont TCP qui est un protocole dit "connecté", et UDP qui est un protocole dit "non connecté".
\subsection{Les structures utilisées généralement}
\subsubsection{sockaddr}
\begin{lstlisting}
struct sockaddr {
  unsigned short sa_family; // address family, AF_xxx
  char sa_data [14];        // 14 bytes of protocol address
};
\end{lstlisting}
\subsubsection{sockaddr\textunderscore{}in}
Plus facile à utiliser.
\begin{lstlisting}
struct sockaddr_in {
  short int sin_family; // Address family
  unsigned short int sin_port; // Port number
  struct in_add r sin_addr; // Internet address
  unsigned char sin_zero[8]; // Same size as struct sockaddr
};
\end{lstlisting}
\subsubsection{in\textunderscore{}addr}
\begin{lstlisting}
struct in_addr {
  unsigned long s_addr; // that's a 32-bit long, or 4 bytes
};
\end{lstlisting}
\subsection{Les conversions}
\begin{description}
  \item htons() "Host to Network Short"
  \item htonl() "Host to Network Long"
  \item ntohs() "Network to Host Short"
  \item ntohl() "Network to Host Long"
\end{description}
Note: dans le struct sockaddr\textunderscore{}in:\newline
les champs sin\textunderscore{}addr et sin\textunderscore{}port doivent aussi être converti.\par
La fonction inet\textunderscore{}aton qui permet d'assigner une valeur au champ sin\textunderscore{}addr d'un struct sockaddr\textunderscore{}in à partir d'un string en notation pointée:
\begin{lstlisting}
#include <sys/socket.h>
#include <netinet/in.h>
#include <arpa/inet.h>

int inet_aton(const char *cp, struct in_addr *inp);
\end{lstlisting}
Exemple de création et d'initialisation d'un struct sockaddr\textunderscore{}in:
\begin{lstlisting}
  struct sockaddr_in my_addr;
  my_addr.sin_family=AF_INET; // host byte order
  my_addr.sin_port = htons (MYPORT); // short, network byte order
  net_aton( "10.12.110.57", &(my_addr.sin_addr));
  memset (&(my_addr.sin_zero), '\0' ,8); // zero the rest
\end{lstlisting}
Variante: fonction inet\textunderscore{}addr(const char* cp), utilisée de la façon suivante:
\begin{lstlisting}
  struct sockaddr_in ina;
  ina.sin_addr.s_addr = inet_addr("10.12.110.57");
\end{lstlisting}
Et inet\textunderscore{}ntoa(struct in\textunderscore{}addr in) fait la conversion inverse.
\subsection{socket}
Permet d'initialiser le socket:
\begin{lstlisting}
#include <sys/types.h>
#include <sys/socket.h>

int socket(int domain, int type, int protocol)
\end{lstlisting}
\begin{enumerate}
  \item le premier paramètre doit être mis à PF\textunderscore{}INET pour les communications sur le réseau.
  \item le deuxième indique le type de socket. Ici, nous utiliserons SOCK\textunderscore{}STREAM
  \item le troisième est généralement mis a 0
\end{enumerate}
\subsection{bind}
Bind permet d'associer un socket avec un numéro de port sur la machine locale. Le numéro de port est utilisé par le kernel pour associer les paquets entrants dans la machine vers un socket descriptor particulier.
\begin{lstlisting}
#include <sys/types.h>
#include <sys/socket.h>

int bind(int sockfd, struct sockaddr *my_addr, int addrlen);
\end{lstlisting}
\begin{enumerate}
  \item Le premier paramètre est le socket descriptor renvoyé par socket().
  \item Le second contient le port de l'adresse IP de la machine locale (soit 0, et l'OS le choisit au hasard soit un nombre compris entre 1024 et 65535 qui n'est pas déjà utilisé).
  \item Quant au troisième paramètre, on peut le mettre à sizeof(struct sockaddr). La constante INADDR\textunderscore{}ANY pour l'adresse IP symbolise l'adresse de la machine locale.
\end{enumerate}
Deux appels successifs à bind() peuvent entraîner des délais d'attente (environ 1 minute) et un message d'erreur. On peut s'en débarrasser grâce à la fonction setsockopt():
\begin{lstlisting}
  int yes =1;
  setsockopt(listener, SOL_SOCKET, SO_REUSEADDR, &yes, sizeof(int));
\end{lstlisting}
\subsection{connect}
Connect permet de se connecter à un ordinateur distant:
\begin{lstlisting}
#include <sys/types.h>
#include <sys/socket.h>

int connect(int sockfd, struct sockaddr *serv_addr ,int addrlen);
\end{lstlisting}
\begin{enumerate}
  \item Le premier paramètre est le socket descriptor renvoyé par socket().
  \item Le second contient le port de l'adresse IP de la machine distante (et non locale comme dans le bind, sinon, le reste c'est comme le bind). (soit 0, et l'OS le choisis au hasard soit un nombre compris entre 1024 et 65535 qui n'est pas déjà utilisé).
  \item Quant au troisième paramètre, on peut le mettre à sizeof(struct sockaddr). La constante INADDR\textunderscore{}ANY pour l'adresse IP symbolise l'adresse de la machine locale.
\end{enumerate}
Un appel à connect() doit toujours être précédé d'un appel à socket(), mais pas forcément d'un appel à bind(). En cas d'absence d'appel à bind(), un numéro de port sera assigné automatiquement à notre socket.
\subsection{listen}
Ce system call est utile pour des processus attendant des connexions de l'extérieur. Il permet d'écouter sur le socket si des ordinateurs distants essayent de se connecter sur notre numéro de port. L'acceptation de connexions se fait au moyen de deux system calls consécutifs: listen() et accept().
\begin{lstlisting}
int listen(int sockfd, int backlog);
\end{lstlisting}
\begin{enumerate}
  \item Le premier paramètre est le socket descriptor renvoyé par socket()
  \item le second est le nombre de demandes de connexions maximum dans la file d'attente. Les demandes de connexion sont mises en file d'attente jusqu'au moment où la machine effectue un accept(). Une valeur typique pour backlog est 20.
\end{enumerate}
Il convient de faire un bind() avant un listen(), sinon le kernel nous associera un port aléatoire, ce qui posera problème ici (connexions distantes).
\subsection{accept}
Des ordinateurs distants vont à présent essayer de se connecter sur notre machine à un port sur lequel nous écoutons (grâce à listen()). Leur demande de connexion sera mise en file d'attente jusqu'à ce que nous les acceptions (une à la fois), par accept():
\begin{lstlisting}
#include <sys/types.h>
#include <sys/socket.h>

int accept(int s, struct sockaddr *addr, int *addrlen);
\end{lstlisting}
\begin{enumerate}
  \item Le premier paramètre est le socket descriptor sur lequel on a fait le listen().
  \item Le second est un pointeur vers un struct sockaddr\textunderscore{}in local qui contiendra des informations sur la machine qui s'est connecté à nous.
  \item Le troisième paramètre est l'adresse d'une variable locale dont la valeur doit être initialisée à sizeof(struct sockaddr\textunderscore{}in).
\end{enumerate}
accept() renvoie un nouveau socket desccriptor qui devra dorénavant être utilisé pour toutes les communications avec la machine qui vient de se connecter.
\subsection{send}
send() permet d'envoyer des informations à un ordinateur distant au travers du socket descriptor associé à la connexion établie avec cet ordinateur:
\begin{lstlisting}
int send(int sackfd, const void *msg, int len, int flag);
\end{lstlisting}
\begin{enumerate}
  \item Le premier paramètre est le socket descriptor renvoyé par socket();
  \item Le deuxième paramètre contient l'information à envoyer;
  \item Le troisième contient sa longueur;
  \item Le quatrième peut être mis à 0.
\end{enumerate}
send() renvoie le nombre de bytes envoyés. En fait, send() envoie le maximum de données possible, mais cela peut encore être inférieur à len. Il faudra alors ré-envoyer le reste.
\subsection{recv}
Recv permet de recevoir des informations envoyées par un ordinateur distant au travers du socket descriptor associé à la connexion établie avec cet ordinateur:
\begin{lstlisting}
int recv(int sockfd, void *buf, int len, unsigned int flags);
\end{lstlisting}
\begin{enumerate}
  \item Le premier paramètre est le socket descriptor renvoyé par socket();
  \item Le deuxième paramètre est un vecteur qui contiendra l'information lue;
  \item Le troisième contient sa longueur;
  \item Le quatrième peut être mis à 0.
\end{enumerate}
Le comportement par défaut de recv() est d'attendre jusqu'à ce qu'une information soit disponible (lecture bloquante).\par
recv() renvoie le nombre de bytes reçus ou 0 si la machine distante a déjà fermé son socket (à l'aide de close(int sockfd)).
\subsection{shutdown}
Shutdown peut être utilisé pour fermer une connexion sur un socket descriptor. Cela interdit toute lecture ou écriture ultérieure sur ce socket.
\begin{lstlisting}
int shutdown(int sockfd, int how);
\end{lstlisting}
\begin{enumerate}
  \item Le premier paramètre est le socket descriptor renvoyé par socket()
  \item Le second paramètre peut être mis à 0 (réceptions interdites), 1 (envois interdits), ou 2 (envois et réceptions interdits).
\end{enumerate}
Un shutdown() doit toujours être suivi d'un close(int sockfd).
\subsection{getpeername}
Getpeername() permet de savoir qui est à l'autre bout d'une connexion par socket
\begin{lstlisting}
#include <sys/socket.h>

int getpeername(int sockfd, struct sockaddr $addr, int *addrlen);
\end{lstlisting}
\begin{enumerate}
  \item Le premier paramètre est le socket descriptor renvoyé par socket()
  \item Le second paramètre contiendra les informations concernant la machine distante.
  \item Le troisième paramètre contiendra la taille de ces informations
\end{enumerate}
Une fois l'adresse obtenue, les fonctions inet\textunderscore{}ntoa() et gethostbyaddr() permettent de l'afficher aisément ou d'obtenir des informations supplémentaires.
\subsection{gethostname}
Gethostname permet de connaître le nom de la machine locale.
\begin{lstlisting}
#include <unistd.h>

int gethostname(char *hostname, size_t size);
\end{lstlisting}
\subsection{gethostbyname}
Cette fonction prend en paramètre un nom de machine et renvoie son adresse IP:
\begin{lstlisting}
#include<netdb.h>

struct hostent *gethostbyname(const char *name );
\end{lstlisting}
Le struct renvoyé est le suivant:
\begin{lstlisting}
struct hostent {
    char *h_name;
    char *h_aliases;
    int h_addrtype;
    int h_length;
    char *h_addr_list; // vecteur d'adresses
  };
  #define h_addr h_addr_list[0]
\end{lstlisting}
\subsubsection{Exemple sans select, correctif de tp}
\begin{lstlisting}
// Serveur.c
#include <stdio.h>
#include <stdlib.h>
#include <unistd.h>
#include <errno.h>
#include <string.h>
#include <sys/types.h>
#include <sys/socket.h>
#include <netinet/in.h>
#include <arpa/inet.h>
#include <sys/wait.h>
#include <signal.h>

#define MYPORT 3490    // the port users will be connecting to

#define BACKLOG 10     // how many pending connections queue will hold

void sigchld_handler(int s) {
    while(wait(NULL) > 0);
}

int main(void) {
    int sockfd, new_fd;  // listen on sock_fd, new connection on new_fd
    struct sockaddr_in my_addr;    // my address information
    struct sockaddr_in their_addr; // connector's address information
    socklen_t sin_size;
    struct sigaction sa;
    int yes = 1;

    if ((sockfd = socket(AF_INET, SOCK_STREAM, 0)) == -1) {
        perror("socket");
        exit(1);
    }
    if (setsockopt(sockfd, SOL_SOCKET, SO_REUSEADDR, &yes, sizeof(int)) == -1) {
        perror("setsockopt");
        exit(1);
    }

    my_addr.sin_family = AF_INET;         // host byte order
    my_addr.sin_port = htons(MYPORT);     // short, network byte order
    my_addr.sin_addr.s_addr = INADDR_ANY; // automatically fill with my IP
    memset(&(my_addr.sin_zero), '\0', 8); // zero the rest of the struct

    if (bind(sockfd, (struct sockaddr *)&my_addr, sizeof(struct sockaddr)) == -1) {
        perror("bind");
        exit(1);
    }
    if (listen(sockfd, BACKLOG) == -1) {
        perror("listen");
        exit(1);
    }
    sa.sa_handler = sigchld_handler; // reap all dead processes
    sigemptyset(&sa.sa_mask);
    sa.sa_flags = SA_RESTART;
    if (sigaction(SIGCHLD, &sa, NULL) == -1) {
        perror("sigaction");
        exit(1);
    }
    while(1) {  // main accept() loop
        sin_size = sizeof(struct sockaddr_in);
        if ((new_fd = accept(sockfd, (struct sockaddr *)&their_addr, &sin_size)) == -1) {
            perror("accept");
            continue;
        }
        printf("server: got connection from %s\n",inet_ntoa(their_addr.sin_addr));
        if (!fork()) { // this is the child process
            close(sockfd); // child doesn't need the listener
            if (send(new_fd, "Hello, world!\n", 14, 0) == -1)
                perror("send");
            close(new_fd);
            exit(0);
        }
        close(new_fd);  // parent doesn't need this
    }
    return 0;
}
\end{lstlisting}
\begin{lstlisting}
// client.c

#include <stdio.h>
#include <stdlib.h>
#include <unistd.h>
#include <errno.h>
#include <string.h>
#include <netdb.h>
#include <sys/types.h>
#include <netinet/in.h>
#include <sys/socket.h>

#define PORT 3490 // the port client will be connecting to

#define MAXDATASIZE 100 // max number of bytes we can get at once

int main(int argc, char *argv[])
{
    int sockfd, numbytes;
    char buf[MAXDATASIZE];
    struct hostent *he;
    struct sockaddr_in their_addr; // connector's address information

    if (argc != 2) {
        fprintf(stderr,"usage: client hostname\n");
        exit(1);
    }
    if ((he=gethostbyname(argv[1])) == NULL) {  // get the host info
        perror("gethostbyname");
        exit(1);
    }
    if ((sockfd = socket(AF_INET, SOCK_STREAM, 0)) == -1) {
        perror("socket");
        exit(1);
    }
    their_addr.sin_family = AF_INET;    // host byte order
    their_addr.sin_port = htons(PORT);  // short, network byte order
    their_addr.sin_addr = *((struct in_addr *)he->h_addr);
    memset(&(their_addr.sin_zero), '\0', 8);  // zero the rest of the struct

    if (connect(sockfd, (struct sockaddr *)&their_addr, sizeof(struct sockaddr)) == -1) {
        perror("connect");
        exit(1);
    }
    if ((numbytes=recv(sockfd, buf, MAXDATASIZE-1, 0)) == -1) {
        perror("recv");
        exit(1);
    }
    buf[numbytes] = '\0';

    printf("Received: %s",buf);

    close(sockfd);

    return 0;
}
\end{lstlisting}
\subsection{select}
Un serveur doit parfois simultanément pouvoir recevoir des données sur un socket et accepter de nouvelles connexions. Cela peut poser problème, car les recv() et les accept() sont bloquants. On peut les rendre non-bloquants, mais on monopoliserait le processeur.\par
Solution : la fonction select() permet d'écouter plusieurs socket descriptors simultanément et d'être réveillé par le premier qui devient actif.
\begin{lstlisting}
#include <sys/time.h>
#include <sys/types.h>
#include <unistd.h>

int select(int n, fd_set *readfds, fd_set *writefds,
               fd_set *exceptfds, struct timeval *timout);
\end{lstlisting}
Le type fd\textunderscore{}set représente un ensemble de file descriptors. Ces ensembles peuvent être mis à jour avec les macros suivantes:
\begin{description}
  \item [FD\textunderscore{}SET(int fd, fd\textunderscore{}set *set)] ajouter fd à un ensemble
  \item [FD\textunderscore{}CLR(int fd, fd\textunderscore{}set *set)] enlever fd d'un ensemble.
  \item [FD\textunderscore{}ZERO(fd\textunderscore{}set *set)] vider un ensemble.
  \item [FD\textunderscore{}ISSET(int fd, fd\textunderscore{}set *set)] tester si fd appartient à l'ensemble.
\end{description}
select() écoute trois ensembles de file descriptors: readfds, writefds, et exceptfds et attend que (au moins) l’un d’entre eux devienne actif
\begin{enumerate}
  \item le premier paramètre (n) doit être initialisé au max des fd + 1.
  \item le second paramètre (readfds) contiendra les fd prêts à être lus l’appel de la fonction.
  \item le troisième paramètre (writefds) contiendra les fd prêts à être écrits l’appel de la fonction.
  \item le quatrième paramètre (timeval) utilisé pour spécifier un timeout, est du type suivant:
  \begin{lstlisting}
    struct timeval {
      int tv_sec;    // secondes
      int tv_usec;   // microsecondes
    };
  \end{lstlisting}
\end{enumerate}
Exemple de select seul: (Tp)
\begin{lstlisting}
#include <stdio.h>
#include <sys/time.h>
#include <sys/types.h>
#include <unistd.h>

int main() {
  fd_set read_set;
  struct timeval timeout;

  FD_SET(STDIN_FILENO, &read_set);
  timeout.tv_sec = 3;  // attend 3 secondes
  timeout.tv_usec = 0;
  select(STDIN_FILENO + 1, &read_set, NULL, NULL, &timeout); //

  if (FD_ISSET(STDIN_FILENO, &read_set)) { // attend une entrée au clavier
    char buffer[1024];
    int end = read(STDIN_FILENO, buffer, 1024);

    buffer[end-1] = '\0'; // buffer[end-1] contient '\n'
    printf("Chaine de caractere lue : %s\n", buffer);
  }
  else {
    printf("Timeout\n");
  }
}
\end{lstlisting}
\section{Sponsors/sources}
Ce document est complètement sponsorisé par OpenClassrooms, developpez.com et certainement plein d'autres que j'ai déjà oublié.\newline
\href{http://emmanuel-delahaye.developpez.com/tutoriels/c/bonnes-pratiques-codage-c/#LI-A}{Liens des bonnes pratiques pour C} (oui, il est déjà apparu en haut)\newline
\href{http://melem.developpez.com/tutoriels/langage-c/initiation-langage-c/}{initiation au language C} Si vous êtes complètement perdu car vous ne savez pas ce que veut dire C++ et que vous n'en avez jamais fait, c'est assez long, je préviens (en même temps, vous venez de loin)\newline
\href{https://c.developpez.com/cours/20-heures/}{Le C en 20h} askip, bon, je ne pense pas que ce soit la référence la plus intéressante\newline
\href{https://c.developpez.com/cours/poly-c/?page=page_1}{Un autre cours du C mais uniquement pour les bases} (mais déjà bien poussé tout en restant sur la base)\newline
\href{https://fr.tuto.com/langage-c/}{3 tuto en vidéo}, je ne les ai pas encore visionné, donc je sais pas vous dire ce que ça vaut, mais je pose ça la quand même\newline
\href{https://openclassrooms.com/courses/apprenez-a-programmer-en-c}{Le cours de OpenClassrooms}, 40 h de cours, si vous avez le temps, ou si vous voulez le parcourir rapidement, mais il ne parle pas de threads, juste les fonctionnalités classiques (pointeur, tableau, etc)
\end{document}
